The motion control is meant to give the robot a smooth and accurate movement that enables wall following and movement through most environments that could appear with the given constraints on the environment.

\subsection{Motor control}

The low level control is performed by the motor control. It receives a velocity message (right speed and left speed) and tries to keep the reference speeds by feeding them into two controllers, one for each wheel. The controllers takes the error (reference speed minus actual speed) and multiplies it with a constant that is tuned manually to obtain a good control performance. This type of controller is called a P-controller (P for proportional). The actual speed is calculated from the time derivative of the wheel encoder measurements and then subtracted from the reference speed to get the error.
With a steady speed the robot can be assumed to either move in a straight line or rotate on the spot depending on the reference signal.

\subsection{Angular and linear movement }

The two basic movements that are needed to explore the whole environment are rotation on the spot and moving straight. Rotation is performed by a P-controller that computes a control signal based on the difference between the reference angle and the actual angle. The angular speed is also saturated to prevent unreasonable high or low speeds. Further, the motion is smoothed by applying an acceleration phase while starting a rotation movement and a lower acceleration results in less wheel slippage and thereby a more accurate final angle. The output (rotation speed) from the rotation function is sent to one motor, and the other gets the rotation speed with switched sign. The architecture of the control system for rotation is shown in figure \ref{fig:rotation_blockdiagram}.

\begin{figure}[h]
\label{fig:rotation_blockdiagram}
    \begin{centering}
   	 \includegraphics[scale=0.5]{figures/rotate_blockdiagram.pdf}
   	 \caption{Controller architecture for rotate function.}\label{fig:rotation_blockdiagram}
    \end{centering}
\end{figure}

The straight movement is based on the same idea but instead of sending a positive speed on one wheel and a negative on the other, both speeds have the same sign to either move the robot backwards or forward. 

\subsection{Wall following}
\label{subsec:wallFollowing}

Wall following moves the robot along a wall and makes it turn if there is any potential collision in front. The wall following node is using the previously mentioned functionalities, linear movement and angular movement, to be able to follow the wall. In addition to the encoder measurements it  uses the IR sensors to determine the distance and an angle to the wall. These measurements have the benefit of having a absolute error compared to the encoder measurements that drift over time and the error integrates and eventually becomes very large.
	The sensor input is sent to a P-controller that controls the angle to the wall, so that the robot drives parallel to the wall. This makes the robot move in a fairly straight line while doing wall following. The distance to the wall is only controlled in case the robot exceeds a threshold of being too close or too far from the wall. This makes the robot behave smoother and do less corrections to keep the right distance and will also improve the odometry since the robot turns less.
	The wall following functionality is based on a state machine that changes state when the wall cannot be followed anymore without further actions. The state machine is shown in figure  \ref{fig:followWallStates} and the states are designed to handle most tricky situations that can appear in the environment. The Conditions used in the state machine are explained in table \ref{tab:conditions}. The states are briefly explained in table \ref{tab:followWallStates}.

\begin{table}[h!]
\caption{Conditions for state transitions}
\centering
  \begin{tabular}{l|l}
    % \hline
    \textbf{Condition} & \textbf{Description} \\ \hline
    rF & Right front IR sensor sees a wall \\ \hline
    rL & Right left IR sensor sees a wall \\ \hline
    ES & Distance to wall error is below threshold \\ \hline
    EB & Distance to wall error is above threshold \\ \hline
    WF & Wall in front detected \\ \hline
    NW & Robot is next to wall \\
    %\hline
  \end{tabular}
  \label{tab:conditions}
\normalsize
\end{table}

\begin{figure}[h]
    \begin{centering}
   	 \includegraphics[scale=0.75]{figures/followWallStates_nice.pdf}
   	 \caption{Blockdiagram showing the state transitions}\label{fig:followWallStates}
    \end{centering}
\end{figure}

\begin{table}[h!]
\centering
  \caption{Descriptions of follow wall states}
  \begin{tabular}{l|p{10cm}}
    % \hline
    \textbf{State} & \textbf{Description} \\ \hline
    Follow Wall & A state that follows the right wall as long as there is a wall on the robot’s right side. \\ \hline
    Rotate right & Rotates 90 degrees right. \\ \hline
    Rotate left & Rotates 90 degrees left. \\ \hline
    Align to wall & Aligns the robot to the right wall. \\ \hline
    Align to front wall & Moves the robot to a certain distance and aligns it to the front wall. \\ \hline
    Handle evil walls & Drives forward for a certain distance. Every time the front IR on the right side sees a wall, this distance is increased. \\ \hline
    Move tail & If e.g. the right wall ends, this state moves the robot forward so that it has some clearance to to go around corners. \\ \hline
    Edge of wall & This state prevents the robot form making quick heading corrections when a wall ends and the IR suddenly changes value. \\ \hline
    Look for beginning of wall & Moves straight until both the front and the back IR see the same wall.  \\ \hline
    Look for end of wall & If the robot is following a wall and the right front IR sensor doesn’t see the wall any more, this state is used to drive the robot straight forward until the right back IR doesn’t see the wall either. \\ \hline
    T intersection handling & If the robot detects a wall in front, this state first aligns the robot to the front wall and then switches to Rotate right to check if it’s possible to go right. \\ 
    \hline
  \end{tabular}
\normalsize
\label{tab:followWallStates}
\end{table}

\subsection{Move to coordinate}
For the path execution described in \ref{subsubsec:pathExec} it is necessary to move freely away from any wall. Move coordinate is used to do this. It is based on the same basic functions as the wall following function is based on, namely the linear movement and the angular movement. As the name tells it moves to a specified coordinate by first rotating and then move straight to that point. The coordinate is given in the robot's local coordinate system.

\subsection{Movement control}
There are some more movements such as aligning to a wall on the right, to a wall in the front or bringing the robot into a position so that it can follow a wall. To integrate all these movements into the system and to provide an easy way to switch between them a node called MovementControl is used.
This node is the only node that sends direct commands to the node MotorControl.

 The above described functionalities are all encapsulated into different C++ classes that derive from one abstract class \textit{MovementState}. The node MovementControl is always in exactly one state which is such a movement. Thus it is guaranteed that multiple movements are not running at the same time which would result into a chaotic movement of the robot. Additionally the encapsulation of the functionality into classes allows to reuse it in other movements. For instance, there is one class that implements the functionality to rotate. Instead of reimplementing this functionality inside the wall follower where a rotation function is needed, it is possible to just create an instance of this rotation class (\textit{MoveRotate}) and use its simple interface to let the robot rotate. That way we could easily reuse all movement components to build more complex and higher level movements such as wall following out of it.   